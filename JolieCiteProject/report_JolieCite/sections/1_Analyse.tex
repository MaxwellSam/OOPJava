\chapter{Analyse}

La ville de \textit{JolieCité} souhaite un programme de gestion de ses évenements culturels. 
La ville dispose de 4 salles dont 3 sont disponibles pour acceuillir des évenements. 
Elle recoit en debut d'année une liste d'évenements qui devront être affectés dans les salles disponibles. 
L'objectif du projet est de développer un programme de gestion qui attribura aux évenements une salle.

\section{Strucure des données}

On distingues deux types d'évenements culturelles : \textbf{les concerts}, qui presentent un nom (le nom du groupe) et une date, et \textbf{les pièces de théatre} qui presentent egalement un nom (le nom de la pièce) mais se distinguent par une liste de dates.

\textbf{Les évenements} ont une capacité de salle demandée. Ce critaire sera un paramétre decisif dans le choix de la salle pour l'évenement culturel.

\textbf{Les salles} presentent une capacité maximal et des horaires d'ouvertures.

\section{Contraintes}

L'affectation des salles au évenements devra prendre en compte la capacité souhaité pour l'évenement. 
La salle doit pouvoir acceillir le nombre de personnes attendu à l'évenement. 

L'affectation devra egalement prendre en compte la disponibilité des salles. 
Il faudra que les horaires de l'évenement correspondant avec les horaires disponnibles de la salle pour que cette dernière lui soit attribuée. 


